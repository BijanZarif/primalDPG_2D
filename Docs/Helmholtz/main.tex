\documentclass{article}

\usepackage{fullpage}
\usepackage{array}
\usepackage{amsmath,amssymb,amsfonts,mathrsfs,amsthm}
\usepackage[utf8]{inputenc}
\usepackage{listings}
\usepackage{mathtools}
\usepackage{pdfpages}
\usepackage[textsize=footnotesize,color=green]{todonotes}
\usepackage{bm}
\usepackage{tikz}
\usepackage[normalem]{ulem}

\usepackage{graphicx}
\usepackage{subfigure}

\usepackage{color}
\usepackage{pdflscape}
\usepackage{pifont}

\usepackage{bibentry}
\nobibliography*

\renewcommand{\topfraction}{0.85}
\renewcommand{\textfraction}{0.1}
\renewcommand{\floatpagefraction}{0.75}

\newcommand{\vect}[1]{\ensuremath\boldsymbol{#1}}
\newcommand{\tensor}[1]{\underline{\vect{#1}}}
\newcommand{\del}{\triangle}
\newcommand{\grad}{\nabla}
\newcommand{\curl}{\grad \times}
\renewcommand{\div}{\grad \cdot}
\newcommand{\ip}[1]{\left\langle #1 \right\rangle}
\newcommand{\eip}[1]{a\left( #1 \right)}
\newcommand{\td}[2]{\frac{d#1}{d#2}}
\newcommand{\pd}[2]{\frac{\partial#1}{\partial#2}}
\newcommand{\pdd}[2]{\frac{\partial^2#1}{\partial#2^2}}

\newcommand{\circone}{\ding{192}}
\newcommand{\circtwo}{\ding{193}}
\newcommand{\circthree}{\ding{194}}
\newcommand{\circfour}{\ding{195}}
\newcommand{\circfive}{\ding{196}}

\newcommand{\Reyn}{\rm Re}

\newcommand{\bs}[1]{\boldsymbol{#1}}
\DeclareMathOperator{\diag}{diag}

\newcommand{\equaldef}{\stackrel{\mathrm{def}}{=}}

\newcommand{\tablab}[1]{\label{tab:#1}}
\newcommand{\tabref}[1]{Table~\ref{tab:#1}}

\newcommand{\theolab}[1]{\label{theo:#1}}
\newcommand{\theoref}[1]{\ref{theo:#1}}
\newcommand{\eqnlab}[1]{\label{eq:#1}}
\newcommand{\eqnref}[1]{\eqref{eq:#1}}
\newcommand{\seclab}[1]{\label{sec:#1}}
\newcommand{\secref}[1]{\ref{sec:#1}}
\newcommand{\lemlab}[1]{\label{lem:#1}}
\newcommand{\lemref}[1]{\ref{lem:#1}}

\newcommand{\mb}[1]{\mathbf{#1}}
\newcommand{\mbb}[1]{\mathbb{#1}}
\newcommand{\mc}[1]{\mathcal{#1}}
\newcommand{\nor}[1]{\left\| #1 \right\|}
\newcommand{\snor}[1]{\left| #1 \right|}
\newcommand{\LRp}[1]{\left( #1 \right)}
\newcommand{\LRs}[1]{\left[ #1 \right]}
\newcommand{\LRa}[1]{\left\langle #1 \right\rangle}
\newcommand{\LRb}[1]{\left| #1 \right|}
\newcommand{\LRc}[1]{\left\{ #1 \right\}}

\newcommand{\Grad} {\ensuremath{\nabla}}
\newcommand{\Div} {\ensuremath{\nabla\cdot}}
\newcommand{\Nel} {\ensuremath{{N^\text{el}}}}
\newcommand{\jump}[1] {\ensuremath{\LRs{\![#1]\!}}}
\newcommand{\uh}{\widehat{u}}
\newcommand{\Bh}{\widehat{B}}
\newcommand{\fnh}{\widehat{f}_n}
\renewcommand{\L}{L^2\LRp{\Omega}}
\newcommand{\pO}{\partial\Omega}
\newcommand{\Gh}{\Gamma_h}
\newcommand{\Gm}{\Gamma_{-}}
\newcommand{\Gp}{\Gamma_{+}}
\newcommand{\Go}{\Gamma_0}
\newcommand{\Oh}{\Omega_h}

\newcommand{\eval}[2][\right]{\relax
  \ifx#1\right\relax \left.\fi#2#1\rvert}

\def\etal{{\it et al.~}}


\def\arr#1#2#3#4{\left[
\begin{array}{cc}
#1 & #2\\
#3 & #4\\
\end{array}
\right]}
\def\vecttwo#1#2{\left[
\begin{array}{c}
#1\\
#2\\
\end{array}
\right]}
\def\vectthree#1#2#3{\left[
\begin{array}{c}
#1\\
#2\\
#3\\
\end{array}
\right]}
\def\vectfour#1#2#3#4{\left[
\begin{array}{c}
#1\\
#2\\
#3\\
#4\\
\end{array}
\right]}

\newcommand{\G} {\Gamma}
\newcommand{\Gin} {\Gamma_{in}}
\newcommand{\Gout} {\Gamma_{out}}


\title{Experiments with primal DPG Helmholtz}
\begin{document}
\section{Variational form}
For purposes of testing, I took the homogeneous Helmholtz equation
\begin{align*}
k^2 u + \Delta u &= 0 \quad \text{ on } \Omega\\
-iku + \pd{u}{n} &= g, \quad \text{ on } \partial\Omega.
\end{align*}
Defining the mesh $\Omega_h$ and mesh skeleton $\Gamma_h$, an integration by parts leads to Jay's variational formulation 
\[
b(u,v) = k^2\LRp{u,v}_{\L} - \LRp{\grad u,\grad v}_{\L} + \LRa{\widehat{\pd{u}{n}},v}_{\Gamma_h}.
\]
In order to incorporate impedance boundary conditions, I took a slightly different approach by adding the boundary term $\LRa{iku,v}_{\Gamma}$ to the variational form and redefining the flux as the impedance condition, such that the variational form is
\[
b(u,v) = k^2\LRp{u,v}_{\L} - \LRp{\grad u,\grad v}_{\L} + \LRa{iku,v}_{\Gamma}+ \LRa{\widehat{-iku + \pd{u}{n}},v}_{\Gamma} + \LRa{\widehat{\pd{u}{n}},v}_{\Gamma_h^0}.
\]
Then, the variational problem for the Helmholtz equation becomes
\[
k^2\LRp{u,v}_{\L} - \LRp{\grad u,\grad v}_{\L} + \LRa{iku,v}_{\Gamma}+ \LRa{\widehat{\pd{u}{n}},v}_{\Gamma_h^0} = \LRa{g,v}_{\Gamma},
\]
which more closely resembles the form of a standard Galerkin problem.  
\section{Primal DPG for Helmholtz}
The test norm is taken to be 
\[
\nor{v}^2_V \coloneqq \alpha\nor{v}_{\L}^2 + \nor{\grad v}_{\L}^2
\]
In Jay's paper, $\alpha = k^2$.  Here, we take $\alpha = 1$.  Empirically, this appears to produce lower $L^2$ errors than $\alpha = k^2$.  The resulting spaces are discretized in the same fashion as in Jay's paper.  

\section{Overlapping additive Schwarz with coarse grid for Helmholtz}



\section{Overlapping multiplicative Schwarz with coarse grid for Helmholtz}


\end{document}